\documentclass[11pt]{article}   % tipo de documento e tamanho das letras

% os seguintes pacotes estendem a funcionalidade básica:
\usepackage[a4paper, total={16cm, 24cm}]{geometry} % tamanho da pagina e do texto
\usepackage[portuguese]{babel}  % traduz para portugues
\usepackage[utf8]{inputenc}
\usepackage{graphicx}           % graficos
\usepackage{amsmath}            % matematica
\usepackage{tikz}               % diagramas
    \usetikzlibrary{shadows}
\usepackage{booktabs}           % tabelas com  melhor aspecto
\usepackage[colorlinks=true]{hyperref}           % links para partes do documento ou para a web
\usepackage{listings}           % incluir codigo
    \renewcommand\lstlistingname{Listagem}  % Listing em portugues
    \lstset{numbers=left, numberstyle=\tiny, numbersep=5pt, basicstyle=\footnotesize\ttfamily, frame=tb,rulesepcolor=\color{gray}, breaklines=true}
\usepackage{blindtext}

% -------------------------------------------------------------------------------------------
\title
{
    \includegraphics[width=0.3\textwidth]{images/logo_universidade.png}
    \\[0.1cm]
    \textbf{Simulador de Escalonamento de Processos} \\
    Sistemas Operativos I
}

\author
{
    \textbf{Professores:} José Saias \\ Pedro Salgueiro
    \textbf{Realizado por:} Miguel de Carvalho (43108) \\ João Pereira (42864) 
}
\date{\today}

% -------------------------------------------------------------------------------------------
%                                Body                                                       %
% -------------------------------------------------------------------------------------------

\begin{document}
\maketitle

% -------------------------------------------------------------------------------------------
\section{Introdução} 

\hspace{0,5cm}Neste trabalho foi solicitado a realização de um programa que simule o \textbf{Escalonamento de Processos} num Modelo de 3 Estados, representado na figura abaixo. \par
\begin{figure}[h!]
    \begin{center}
        \includegraphics[width=0.5\textwidth]{images/states.png}
        \caption{Modelo de 3 Estados}
    \end{center}
\end{figure}
O \textbf{Escalonador de Processos} faz parte do \textbf{Sistema Opertivo} e é responsável por decidir em que momento cada processo estará no CPU. 
Existem muitos algoritmos de escalonamento para realizar essa decisão. \par
Neste trabalho serão utilizados o algoritmo \textbf{FCFS} e o \textbf{Round Robin (RR)}:
\begin{itemize}
    \item O \textbf{FCFS} é um algoritmo de escalonamento não preemptivo que prioriza os processos pela ordem de chegada. Executa o processo todo do inicio ao fim sem o interromper, até estar concluído. Quando aparece um novo processo e ainda existe um em execução, esse novo fica em fila de espera.
    \item O \textbf{Round Robin (RR)} é um algoritmo de escalonamento preemptivo que apresenta um funcionamento igual ao do \textbf{FCFS}, mas com tempo limite de execução, o \textbf{Quantum}. Assim, quando o processo se encontra em execução este será interrompido quando o tempo de execução for igual ao \textbf{Quantum} e fica em fila de espera (\textbf{READY}). 
\end{itemize}

% -------------------------------------------------------------------------------------------
\section{Implementação}

\hspace{0,5cm}

-Tema do site: loja de roupa

-Ficheiros HTML:

    *advanced_search - pagina que permite aos utilizadores realizar pesquisas avançadas de produtos.

    *product - apresentação de um unico produto com descrição
    
    *index - pagina inicial que mostra os varios produtos, categorias e informações do site

- Ficheiro CSS: style.css

- Estrutura do ficheiro index.html:

    *cabeçalho: permite aos utilizadores pesquisar produtos;
    *categorias: lista as várias categorias de produtos;
    *produtos: lista os vários produtos disponíveis na loja online, ou os produtos resultantes de uma pesquisa;
    *about: informação genérica sobre a página.

    Comportamneto responsivo:

- Estrutura do ficheiro product.html:

    *cabeçalho: permite aos utilizadores pesquisar produtos;
    *categorias: lista as várias categorias de produtos;
    *detalhes do produto: descrição do produto com imagem;
    *about: informação genérica sobre a página.

    Comportamneto responsivo:

- Estrutura do ficheiro advanced_search.html:

% -------------------------------------------------------------------------------------------
\section{Funções}
    
\begin{itemize}
    \item Funções pertencentes à manipulação das \verb|queues|:
        \begin{itemize}
            \item A função "\verb|queue_t *create_queue (int sz)|" cria a estrutura e aloca espaço para a \verb|queue|;
            \item A função "\verb|void insert (queue_t *queue, int element)|" adiciona elementos na \textbf{stack};
            \item A função "\verb|bool full (queue_t *queue)|" verifica se a \textbf{stack} se encontra \textbf{cheia}, e caso esteja, devolve \verb|True|;
            \item A função "\verb|bool empty (queue_t *queue)|" verifica se a \textbf{stack} se encontra \textbf{vazia}, e caso esteja, devolve \verb|True|;
            \item A função "\verb|int get (queue_t *queue)|" remove o primeiro elemento da \textbf{stack}, dando \verb|return| desse valor e move os elementos para a esquerda \verb|queue|. Por exemplo, o elemento da posição 2 passa para a posição 1;
            \item A função "\verb|void printQueue(queue_t *queue)|" dá \verb|print| de todos os elementos da \verb|queue|;
            \item A função "\verb|int top(queue_t *queue)|" dá \verb|return| do valor da última posição do \verb|queue|.
        \end{itemize}
    \newpage
    \item Funções pertencentes à manipulação da \verb|struct de processos|:
        \begin{itemize}
            \item A função "\verb|process_t *create_process(int sz)|" cria a estrutura e aloca espaço para o \verb|processo|;
            \item A função "\verb|process_t *insert_process(int beg, int end, int queues, int arr[])|" \newline insere um \verb|processo|;
            \item A função "\verb|int find_PID(int PID, process_t *process_arr[], int n_process)|" procura o \textbf{PID} no \textbf{array} e devolve a sua posição;
            \item A função "\verb|void update_run(int n_process, process_t *process_arr[])|" decrementa a primeira posição do \textbf{run} do \verb|processo|;
            \item A função "\verb|void update_blocked(int n_process, process_t *process_arr[])|" decrementa a primeira posição do \textbf{blocked} do processo;
            \item A função "\verb|void update_index_run(int n_process, process_t *process_arr[], int size)|" \newline move os elementos do \textbf{run} do \verb|processo| para a esquerda e verifique se existe algum número maior que 1000, caso exista irá mudar esse valor para 0;
            \item A função "\verb|void update_index_blocked(int n_process, process_t *process_arr[], int size)| \newline move os elementos do \textbf{blocked} do processo para a esquerda e verifica se existe algum número maior que 1000, caso exista irá mudar esse valor para 0"
        \end{itemize}
\end{itemize}

%-------------------------------------------------------------------------------------------
\section{Execução}

\hspace{0,5cm}Para executar o \textbf{Simulador de Escalonamento} o utilizador deverá passar como argumento o algoritmo a usar e o ficheiro de \textbf{input}.
\begin{itemize}
    \item Caso o utilizador escolha o \textbf{FCFS}, o \textbf{QUANTUM} irá ter um valor de \textbf{999}, deste modo nunca irá obrigar o \verb|processo| a passar para o estado \textbf{READY}.
    \item Caso o utilizador escolha o \textbf{Round Robin}, o programa irá correr com o \textbf{Quantum} definido no código (\verb|#define QUANTUM_RR|).
\end{itemize}
\par Exemplos:
\begin{itemize}
    \item \verb|./.compiled --fcfs input1.txt| irá executar o \textbf{Simulador de Escalonamento} utilizando o algoritmo \textbf{FCFS} com o \verb|input1.txt|
    \item \verb|./.compiled --rr input1.txt| irá executar o \textbf{Simulador de Escalonamento} utilizando o algoritmo \textbf{Round Robin} com o \verb|input1.txt|
\end{itemize}
% ----------\item ---------------------------------------------------------------------------------
\newpage
\section{Análise de Resultados} % Conclusão
\begin{itemize}
    \item Escalonamento com o Algoritmo \textbf{FCFS}:
    \begin{itemize}
        \item Com o \verb|input1.txt|:
        \begin{verbatim}
Instant  0 - Ready: 101           | Run: 100   | Blocked: Empty!
Instant  1 - Ready: 200 300       | Run: 101   | Blocked: 100 
Instant  2 - Ready: 200 300       | Run: 101   | Blocked: 100 
Instant  3 - Ready: 200 300       | Run: 101   | Blocked: 100 
Instant  4 - Ready: 200 300 100   | Run: 101   | Blocked: Empty!
Instant  5 - Ready: 300 100       | Run: 200   | Blocked: 101 
Instant  6 - Ready: 300 100       | Run: 200   | Blocked: 101 
Instant  7 - Ready: 100           | Run: 300   | Blocked: 101 200 
Instant  8 - Ready: 100           | Run: 300   | Blocked: 101 200 
Instant  9 - Ready: 100 101       | Run: 300   | Blocked: 200 
Instant 10 - Ready: 100 101       | Run: 300   | Blocked: 200 
Instant 11 - Ready: 100 101       | Run: 300   | Blocked: 200 
Instant 12 - Ready: 100 101 200   | Run: 300   | Blocked: Empty!
Instant 13 - Ready: 100 101 200   | Run: 300   | Blocked: Empty!
Instant 14 - Ready: 101 200       | Run: 100   | Blocked: 300 
Instant 15 - Ready: 101 200       | Run: 100   | Blocked: 300 
Instant 16 - Ready: 101 200       | Run: 100   | Blocked: 300 
Instant 17 - Ready: 101 200       | Run: 100   | Blocked: 300 
Instant 18 - Ready: 101 200       | Run: 100   | Blocked: 300 
Instant 19 - Ready: 101 200       | Run: 100   | Blocked: 300 
Instant 20 - Ready: 101 200 300   | Run: 100   | Blocked: Empty!
Instant 21 - Ready: 101 200 300   | Run: 100   | Blocked: Empty!
Instant 22 - Ready: 101 200 300   | Run: 100   | Blocked: Empty!
Instant 23 - Ready: 101 200 300   | Run: 100   | Blocked: Empty!
Instant 24 - Ready: 200 300       | Run: 101   | Blocked: 100 
Instant 25 - Ready: 200 300       | Run: 101   | Blocked: 100 
Instant 26 - Ready: 300           | Run: 200   | Blocked: 100 
Instant 27 - Ready: 100           | Run: 300   | Blocked: 200 
Instant 28 - Ready: Empty!        | Run: 100   | Blocked: 200 
Instant 29 - Ready: 200           | Run: 100   | Blocked: Empty!
Instant 30 - Ready: 200           | Run: 100   | Blocked: Empty!
Instant 31 - Ready: 200           | Run: 100   | Blocked: Empty!
Instant 32 - Ready: 200           | Run: 100   | Blocked: Empty!
Instant 33 - Ready: 200           | Run: 100   | Blocked: Empty!
Instant 34 - Ready: Empty!        | Run: 200   | Blocked: Empty!
Instant 35 - Ready: Empty!        | Run: 200   | Blocked: Empty!
Instant 36 - Ready: Empty!        | Run: 200   | Blocked: Empty!    
    
        \end{verbatim}
    \end{itemize}
    \newpage
    \item Escalonamento com o Algoritmo \textbf{RR (Round Robin)}:
    \begin{itemize}
        \item Com o \verb|input1.txt|:
        \begin{verbatim}
Instant  0 - Ready: 101           | Run: 100   | Blocked: Empty!
Instant  1 - Ready: 200 300       | Run: 101   | Blocked: 100 
Instant  2 - Ready: 200 300       | Run: 101   | Blocked: 100 
Instant  3 - Ready: 200 300       | Run: 101   | Blocked: 100 
Instant  4 - Ready: 300 100 101   | Run: 200   | Blocked: Empty!
Instant  5 - Ready: 300 100 101   | Run: 200   | Blocked: Empty!
Instant  6 - Ready: 100 101       | Run: 300   | Blocked: 200 
Instant  7 - Ready: 100 101       | Run: 300   | Blocked: 200 
Instant  8 - Ready: 100 101       | Run: 300   | Blocked: 200 
Instant  9 - Ready: 101 300       | Run: 100   | Blocked: 200 
Instant 10 - Ready: 101 300       | Run: 100   | Blocked: 200 
Instant 11 - Ready: 101 300 200   | Run: 100   | Blocked: Empty!
Instant 12 - Ready: 300 200 100   | Run: 101   | Blocked: Empty!
Instant 13 - Ready: 200 100       | Run: 300   | Blocked: 101 
Instant 14 - Ready: 200 100       | Run: 300   | Blocked: 101 
Instant 15 - Ready: 200 100       | Run: 300   | Blocked: 101 
Instant 16 - Ready: 100 300       | Run: 200   | Blocked: 101 
Instant 17 - Ready: 300 101       | Run: 100   | Blocked: 200 
Instant 18 - Ready: 300 101       | Run: 100   | Blocked: 200 
Instant 19 - Ready: 300 101 200   | Run: 100   | Blocked: Empty!
Instant 20 - Ready: 101 200 100   | Run: 300   | Blocked: Empty!
Instant 21 - Ready: 200 100       | Run: 101   | Blocked: 300 
Instant 22 - Ready: 200 100       | Run: 101   | Blocked: 300 
Instant 23 - Ready: 100           | Run: 200   | Blocked: 300 
Instant 24 - Ready: 100           | Run: 200   | Blocked: 300 
Instant 25 - Ready: 100           | Run: 200   | Blocked: 300 
Instant 26 - Ready: Empty!        | Run: 100   | Blocked: 300 
Instant 27 - Ready: 300           | Run: 100   | Blocked: Empty!
Instant 28 - Ready: 300           | Run: 100   | Blocked: Empty!
Instant 29 - Ready: 100           | Run: 300   | Blocked: Empty!
Instant 30 - Ready: Empty!        | Run: 100   | Blocked: Empty!
Instant 31 - Ready: Empty!        | Run: Empty!| Blocked: 100 
Instant 32 - Ready: Empty!        | Run: Empty!| Blocked: 100 
Instant 33 - Ready: Empty!        | Run: Empty!| Blocked: 100 
Instant 34 - Ready: Empty!        | Run: 100   | Blocked: Empty!
Instant 35 - Ready: Empty!        | Run: 100   | Blocked: Empty!
Instant 36 - Ready: Empty!        | Run: 100   | Blocked: Empty!
Instant 37 - Ready: Empty!        | Run: 100   | Blocked: Empty!
Instant 38 - Ready: Empty!        | Run: 100   | Blocked: Empty!
Instant 39 - Ready: Empty!        | Run: 100   | Blocked: Empty!           
        \end{verbatim}
    \end{itemize}
\end{itemize}

% -------------------------------------------------------------------------------------------
\newpage
\section{Conclusão} % Conclusão
\hspace{0,5cm}Em suma, com a realização deste trabalho "Simulador de Escalonamento" fiquei muito mais esclarecido sobre o seu funcionamento. \par
Saliento que me ajudou a entender como funciona o escalonador e as condições que cada algoritmo (\textbf{FCFS}/\textbf{RR}), usa para proceder à mudança dos processos entre os estados e as respetivas diferenças de tempo no mesmo conjunto de processos, entre os respetivos algoritmos.  
% -------------------------------------------------------------------------------------------
\end{document}